\clearpage

\chapter{\textbf{Métodos}}\label{methods}

\section{Inicio de um capítulo}\label{met:chapters}
In the two-sided layout a new chapter always begins on a right side. Therefore, empty pages will be generated automatically at the end of the previous chapter if necessary. The one-sided layout does not do this. If you want an additional empty page, you have to insert it manually.

\section{Tabelas grandes}\label{met:table}
Tables are a bit complex to generate in latex, a big help are table generators like: \url{https://www.tablesgenerator.com/} . Small tables fit within the text flow. \\
\begin{table} [ht]
    \begin{center}
    %\centering
         \caption[Small table]{Material needed for gels
         }
\begin{tabular}{|l|C{3cm}|C{3cm}|}
\hline
material (conc.) & volume per gel & conc. in gel\\ \hline
Chemical 1 (40~mg/mL)             & 15~\textmu L     & 3.00~ \textmu g/mL\\ \hline
TBS                             & 15~\textmu L     & ---\\ \hline
Chemical 2         & 15~\textmu L     & 3.75~\textmu g/mL\\ \hline
Chemical 3   & 27.5~\textmu L   & 5.7x10$^5$ cells\\ \hline
Chemical 4  & 27.5~\textmu L   & 5.7x10$^5$ cells\\ \hline \hline 
Chemical 5      & 100~\textmu L    & 5.00 mg/mL \\ \hline

        \end{tabular}
 \label{tabfibrin}
    \end{center}
\end{table}
Bigger tables can be put on separate pages. \\


\begin{table} [p] %p makes to table to show up on a separate page. This is useful when the table is really big or two tables should be aligned underneath each other on a separate page.

    \begin{center}
    \caption[FACS analysis]{A lot of complex columns and rows
         }
\begin{tabular}{|l|l|l|l|l|l|l|l|l|l|l|}
\hline
\multicolumn{2}{|c|}{Tube 1}                                 &                       & \multicolumn{2}{c|}{Tube 2}                                 &  & \multicolumn{2}{c|}{Tube 3}                                 & \multicolumn{1}{c|}{} & \multicolumn{2}{c|}{Tube 4}                                 \\ \cline{1-2} \cline{4-5} \cline{7-8} \cline{10-11} 
\multicolumn{1}{|c|}{Antibody} & \multicolumn{1}{c|}{Volume} & \multicolumn{1}{c|}{} & \multicolumn{1}{c|}{Antibody} & \multicolumn{1}{c|}{Volume} &  & \multicolumn{1}{c|}{Antibody} & \multicolumn{1}{c|}{Volume} & \multicolumn{1}{c|}{} & \multicolumn{1}{c|}{Antibody} & \multicolumn{1}{c|}{Volume} \\ \cline{1-2} \cline{4-5} \cline{7-8} \cline{10-11} 
AB1   & 20~\textmu L   &    & AB2    & 20~\textmu L  &  & AB3    & 20~\textmu L    &     & AB4      &20~\textmu L       \\ 
\cline{1-2} \cline{4-5} \cline{7-8} \cline{10-11}  
APC   & MSC-   &  & FITC  & MSC- &  & PE   & MSC-  & & PE   & MSC-    \\
\cline{1-2} \cline{4-5} \cline{7-8} \cline{10-11} 
\multicolumn{1}{|c|}{Antibody} & \multicolumn{1}{c|}{Volume} & \multicolumn{1}{c|}{} & \multicolumn{1}{c|}{Antibody} & \multicolumn{1}{c|}{Volume} &  & \multicolumn{1}{c|}{Antibody} & \multicolumn{1}{c|}{Volume} & \multicolumn{1}{c|}{} & \multicolumn{1}{c|}{Antibody} & \multicolumn{1}{c|}{Volume}\\ 
\cline{1-2} \cline{4-5} \cline{7-8} \cline{10-11} 
AB5  & 2~\textmu L   &   & AB6   & 5~\textmu L  &  & AB7   & 5~\textmu L         &    & AB8    & 5~\textmu L    \\ 
\cline{1-2} \cline{4-5} \cline{7-8} \cline{10-11} 
FITC  & MSC+  &   & APC  & MSC+   &  & \scriptsize{PerCP-Cy5.5}  & MSC+  &  & \scriptsize{PerCP-Cy5.5}  & MSC-       \\ 
\hline
\end{tabular}
 \label{tabfacsmsc}
    \end{center}
\end{table}

\begin{table} [p]
    \begin{center}
    %\centering
         \caption[Another list of antibodies]{Some more antibodies for FACS analysis
         }
\begin{tabular}{|l|l|l|l|l|l|l|l|l|l|l|}
\hline
\multicolumn{2}{|c|}{Tube 1}                                 &                       & \multicolumn{2}{c|}{Tube 2}                                 &  & \multicolumn{2}{c|}{Tube 3}                                 & \multicolumn{1}{c|}{} & \multicolumn{2}{c|}{Tube 4}                                 \\ \cline{1-2} \cline{4-5} \cline{7-8} \cline{10-11} 
\multicolumn{1}{|c|}{Antibody} & \multicolumn{1}{c|}{Volume} & \multicolumn{1}{c|}{} & \multicolumn{1}{c|}{Antibody} & \multicolumn{1}{c|}{Volume} &  & \multicolumn{1}{c|}{Antibody} & \multicolumn{1}{c|}{Volume} & \multicolumn{1}{c|}{} & \multicolumn{1}{c|}{Antibody} & \multicolumn{1}{c|}{Volume} \\ \cline{1-2} \cline{4-5} \cline{7-8} \cline{10-11} 
AB1   & 20~\textmu L   &    & AB2    & 20~\textmu L  &  & AB3    & 5~\textmu L    &     & AB4      & 5~\textmu L       \\ 
\cline{1-2} \cline{4-5} \cline{7-8} \cline{10-11} 
FITC   & EC+   &  & FITC  & EC+ &  & FITC   & EC+  & & FITC   & EC-    \\
\cline{1-2} \cline{4-5} \cline{7-8} \cline{10-11} 
\multicolumn{1}{|c|}{Antibody} & \multicolumn{1}{c|}{Volume} & \multicolumn{1}{c|}{} & \multicolumn{1}{c|}{Antibody} & \multicolumn{1}{c|}{Volume} &  & \multicolumn{1}{c|}{Antibody} & \multicolumn{1}{c|}{Volume} & \multicolumn{1}{c|}{} & \multicolumn{1}{c|}{Antibody} & \multicolumn{1}{c|}{Volume}\\ 
\cline{1-2} \cline{4-5} \cline{7-8} \cline{10-11} 
AB5  & 20~\textmu L   &   & AB6   & 2~\textmu L  &  & AB7   & 20~\textmu L         &    & AB8    & 20~\textmu L    \\ 
\cline{1-2} \cline{4-5} \cline{7-8} \cline{10-11} 
APC  & EC-  &   & APC  & EC-   &  & APC  & EC+  &  & APC  & EC+       \\ 
\hline
\end{tabular}
 \label{tabfacsec}
    \end{center}
\end{table}

\blindtext
\section{Isto é uma secção}
\subsection{Isto é uma subsecção}
\subsubsection{Isto é uma subsubsecção que não aparece no índice}\label{subsub}
Here we have some random equations: 
\begin{equation}\label{Volume}
V = d^2\cdot \frac{\pi}{4}\cdot L
\end{equation}

\begin{equation}\label{Surface}
S = d^2 \cdot \frac{\pi}{4} + \pi \cdot d \cdot L \thickapprox \pi \cdot d \cdot L
\end{equation}

\begin{equation}\label{Volume1}
V = \frac{S^2}{\pi ^2 \cdot L^2} \cdot \frac{\pi}{4}\cdot L = \frac{S^2}{4\pi \cdot L}
\end{equation}
\begin{equation}\label{Length}
L = \frac{S^2}{4\pi \cdot V}
\end{equation}

\begin{equation}\label{stdv}
 s = \sqrt{\frac{\sum_{i=1}^n (x_i-\bar{x})^2}{(n-1)}}
\end{equation}

\newpage
\subsubsection{Outro tipo de tabelas}\label{outrotabelas}
\begin{table} [h!]
	\begin{center}
    %\centering
       
	\begin{tabular}{lcc}
	\toprule
	Indicador & Descrição & Unidade \\
	\midrule
	E1 & $\frac{\text{Custo Total de Manutenção}}{\text{Valor de substituição dos activos}}$ & $\times 100\%$ \\
	E2 & $\frac{\text{Custo Total de Manutenção}}{\text{Valor valor acrescentado + Custos Externos de Manutenção}}$ & $\times 100\%$ \\
	E3 & $\frac{\text{Custo Total de Manutenção}}{\text{Quantidade de \textit{Output}}}$ & \euro / Un. \\
	E5 & $\frac{\text{Custo Total de Manutenção + Custo de inoperação por manutenção}}{\text{Quantidade de \textit{Output}}}$ & \euro / Un. \\
	E7 & $\frac{\text{Valor médio do Stock de Materiais de Manutenção}}{\text{Valor de substituição dos activos}}$ & $\times 100\%$ \\
	E8 & $\frac{\text{Custo Total do pessoal interno de Manutenção}}{\text{Valor de substituição dos activos}}$ & $\times 100\%$ \\
	E9 & $\frac{\text{Custo Total do pessoal externo de Manutenção}}{\text{Valor de substituição dos activos}}$ & $\times 100\%$ \\
	E10 & $\frac{\text{Custo Total dos Serviços de Terceiros}}{\text{Valor de substituição dos activos}}$ & $\times 100\%$ \\
	E11 & $\frac{\text{Custo Total de Materiais de Manutenção}}{\text{Valor de substituição dos activos}}$ & $\times 100\%$ \\
	E12 & $\frac{\text{Custo Total de Materiais de Manutenção}}{\text{Valor médio do stock de Materiais de Manutenção}}$ & $\text{rotações / ano}$ \\
	E15 & $\frac{\text{Custo de Manutenção Correctiva}}{\text{Valor médio do stock de Materiais de Manutenção}}$ & $\times 100\%$ \\
	E16 & $\frac{\text{Custo de Manutenção Preventiva}}{\text{Valor médio do stock de Materiais de Manutenção}}$ & $\times 100\%$ \\
	E19 & $\frac{\text{Custo de Manutenção de Melhoria}}{\text{Valor médio do stock de Materiais de Manutenção}}$ & $\times 100\%$ \\

	\bottomrule

	\end{tabular}
    \caption[Lista (não extensiva) de indicadores-chave económicos]{Lista (não extensiva) de indicadores-chave económicos}              
	\label{tab:indicadore_economicos}
    \end{center}
\end{table}



\addtocontents{toc}{\vspace{0.8cm}}